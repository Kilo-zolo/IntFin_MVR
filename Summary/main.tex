\documentclass{article}
\usepackage[utf8]{inputenc}

\title{Summary of Trading\\
        International Finance}

\author{Krish Gupta s3865829}   

\usepackage{natbib}
\usepackage{graphicx}
\usepackage{booktabs}
\usepackage{placeins}

\let\Oldsubsection\subsection
\renewcommand{\subsection}{\FloatBarrier\Oldsubsection}

\let\Oldsubsubsection\subsubsection
\renewcommand{\subsubsection}{\FloatBarrier\Oldsubsubsection}


\begin{document}
\maketitle

\tableofcontents
\pagebreak

\section*{Summary of Transactions}
\addcontentsline{toc}{section}{Summary of Transactions}
My trading strategy aimed to capitalize on the anticipated strength of the U.S. Dollar (USD), Euro (EUR), and British Pound (GBP) while mitigating exposure to the Australian Dollar (AUD) and Japanese Yen (JPY). The goal was to liquidate AUD and JPY holdings to accumulate USD, which would then be used to invest in EUR and GBP. Each transaction was planned to be significant, capped at USD 10,000,000, to optimize the profit potential.
Initially, I successfully liquidated my AUD holdings, converting them into USD as planned. However, the strategy faced a hitch when it came to JPY liquidation. While I initiated several transactions to sell JPY in exchange for USD, I could not sell as much JPY as I had initially intended. This constraint was primarily due to the limited 2-hour trading window, which hindered the complete execution of my strategy.
I used the accumulated USD to invest in EUR and GBP, with a heavier focus on GBP, which aligned with my analysis that indicated higher profit potential for GBP. However, the time constraints also limited my ability to buy as much GBP as I had initially targeted. Consequently, I ended the session with a portfolio that, while partially aligned with my intended asset allocation, needed to be fully optimized.\\

\noindent In terms of performance, the portfolio registered a slight gain of 0.021810, based on yesterday's (25/08/23) WM Rate. However, a decline was noted based on today's (26/08/23) rate, with the portfolio showing a loss of -0.38685. This volatility underscores the sensitivity of the portfolio to short-term currency fluctuations and the importance of timely execution in optimizing returns.


\section*{Reflection}
\addcontentsline{toc}{section}{Reflection}
The primary issue during the trading session was insufficient time to execute my strategy fully. While the initial plan was robust, the 2-hour window was restrictive. In future sessions, prioritizing transactions based on their impact on portfolio returns could prove beneficial. For example, if GBP continues showing higher profit potential, liquidating JPY holdings to acquire more GBP should take precedence.
Risk management is another area that warrants attention. The strategy involved executing transactions of substantial volume, which, while maximizing profit potential, also exposed the portfolio to significant risk due to market volatility. Implementing a contingency plan or stop-loss measures could mitigate these risks. These tools provide a safety net, mainly when dealing with more volatile or less liquid currencies.
Additionally, the portfolio could benefit from diversification. While the strategy focused on four major currencies, diversifying into other currencies or assets could serve as a hedge against unfavourable market movements in any single currency. This is especially pertinent given the portfolio's sensitivity to short-term exchange rate fluctuations.
Time management needed to be improved. The inability to execute all planned transactions within the limited trading session impacted the portfolio's performance. Future strategies could incorporate time management plans and leverage automated trading systems to ensure better execution within tight time frames.
Finally, the portfolio's performance highlighted the importance of market timing. The swing from a marginal gain to a significant loss within a single day indicates that short-term market trends should be closely monitored to time transactions more effectively.\\

\noindent In conclusion, while the trading strategy was theoretically sound and well-intentioned, its complete execution could have been improved by time constraints and market volatility. The experience has provided me with valuable insights into risk management, time management, and the potential benefits of diversification and market timing.


\end{document}